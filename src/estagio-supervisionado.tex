\chapter{Estágio supervisionado}
\label{cap:estagio-supervisionado}

Neste capítulo, a empresa concedente Avenue Code e o Programa de Estágio Jedi Internship são apresentados. Tópicos envolvendo a história, especialidade e iniciativas da empresa, bem como a estrutura do programa de estágio são ilustrado nas seções a seguir.

\section{Sobre a empresa: Avenue Code}
\label{sec:sobre-a-empresa}

A Avenue Code Desenvolvimento e Comércio de Softwares Ltd é uma empresa consultora de softwares especializada no ramo de \textit{e-commerce} da indústria varejista. Fundada em 2008 pelo CEO Zeo Solomon na cidade de San Francisco (Califórnia, Estados Unidos da América), a Avenue Code atendia clientes americanos, abrindo seu primeiro escritório no Brasil somente um ano depois, na cidade de Belo Horizonte. Nos anos seguintes, a empresa expande e inaugura um escritório na cidade de São Paulo, além de adicionar Amir Razmara e Chase Hill ao time de CEOs. Em 2017, a Avenue Code é formada por mais de 230 consultores em sua equipe, além de inaugurar um quarto escritório, agora na cidade de Nova York~\cite{ac-who-we-are-2017}.

A Avenue Code é especialista no desenvolvimento e utilização de diversos tipos de tecnologias da área de Computação, como aplicações Web e móveis, automação de infraestruturas, sistemas de \textit{backend}, implementações de plataformas, \textit{coaching} Agile e DevOps e integrações corporativas. A Metodologia Ágil, oriunda do Manifesto Ágil para o Desenvolvimento de Software~\cite{agile-2001}, é altamente aplicada na empresa, que busca maximizar sua eficiência através da utilização de princípios Agile no desenvolvimento de projetos~\cite{ac-what-we-do-2017}.

Prezando tanto pela qualidade da tecnologia utilizada quanto pelo ambiente de trabalho dos consultores, a Avenue Code possui uma gama de empresas parceiras e de prêmios obtidos em sua história. Dentre as empresas de tecnologia parceiras da Avenue Code, figuram a Mulesoft~\footnote{MuleSoft: Integration platform for connecting SaaS and enterprise applications. Disponível em: \url{https://www.mulesoft.com/}}, SAP Hybris~\footnote{SAP Hybris: E-commerce solutions. Disponível em: \url{https://www.hybris.com/en/}}, CHEF~\footnote{Chef: automate IT infrastructure. Disponível em: \url{https://www.chef.io/chef/}}, Oracle~\footnote{Oracle: Integrated cloud applications and platform services. Disponível em: \url{https://www.oracle.com/}}, Amazon Web Services~\footnote{Amazon Web Services (AWS): Cloud computing services. Disponível em: \url{https://aws.amazon.com/}} e Adobe~\footnote{Adobe: Creative, marketing and document management solutions. Disponível em \url{www.adobe.com/}}~\cite{ac-partners-2017}. Já entre os prêmios conquistados pela empresa, estão o reconhecimento pelo LoveMondays~\footnote{Love Mondays: A empresa ideal, avaliada por profissionais como você. Disponível em: \url{https://www.lovemondays.com.br/}}, InfoMoney~\footnote{InfoMoney: Notícias, ações e muito mais sobre investimentos. Disponível em: \url{www.infomoney.com.br/}} e San Francisco Business Times's Fast 100~\footnote{San Francisco Business Time Fast 100. Disponível em: \url{http://www.bizjournals.com/sanfrancisco/blog/2016/10/bay-area-fast-growing-private-companies-fast-100.html}} em 2016, além de ser agraciada como uma das Melhores Empresas para Trabalhar~\footnote{Great Place to Work. Disponível em: \url{www.greatplacetowork.com.br/}} em 2016 (\textit{Great Place to Work})~\cite{ac-who-we-are-2017}.

Por fim, a Avenue Code participa de iniciativas de inclusão digital por meio do programa AC Social, que oferta aulas de introdução à tecnologias e computação em escolas carentes. A empresa também oferece cursos ministrados pelos próprios consultores por meio do AC Community, além de ofertar dois programas de estágios distintos, o AC Wonder Women e o Jedi Internship, que será apresentado na seção seguinte~\cite{ac-academy-2017}.

\section{Sobre o estágio: Programa de Estágio Jedi Internship}
\label{sec:sobre-o-estagio}

O Programa de Estágio Jedi Internship consiste de uma rotação (\textit{job rotation}) por cinco áreas de diferentes tecnologias presentes na área de Computação. São elas:

\begin{itemize}
	\item Conceitos e \textit{frameworks} de \textit{user interface} (UI) e \textit{user experience} (UX)
	\item Conceitos e \textit{frameworks} de garantia de qualidade de software
	\item Linguagem de programação Java
	\item Linguagem de programação Ruby
	\item \textit{Framework} .NET
\end{itemize}

Cada área dura cerca de três meses, e em cada uma delas o estagiário aprende conceitos introdutórios e avançados da tecnologia, aplicando-os em projetos internos da empresa. Cada área possui consultores que atuam como mentores para os participantes e, ao final da área, o estagiário elabora um \textit{workshop} para ser apresentado tanto para seus mentores quanto para os outros participantes do Programa, exibindo conceitos, aplicações desenvolvidas e desafios encontrados ao longo dos três meses.

Esse trabalho apresentará os conceitos aprendidos e aplicações desenvolvidas pelo autor ao longo das áreas de Java e UI/UX, sendo que uma fundamentação teórica para ambas as áreas é exibida no capítulo seguinte.