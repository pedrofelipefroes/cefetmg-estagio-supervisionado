\chapter{Introdução}
\label{cap:introducao}

O presente trabalho é um relatório da disciplina de Estágio Supervisionado pertencente ao curso de Engenharia de Computação do Centro Federal de Educação Tecnológica de Minas Gerais (CEFET-MG), realizado pelo discente Pedro Felipe Froes Silva na empresa Avenue Code Desenvolvimento e Comércio de Software Ltda. Este relatório reflete o período de Março a Maio de 2017 durante o estágio do aluno na empresa, contemplando parte do Programa de Estágio Jedi Internship.

O Programa de Estágio Jedi Internship da Avenue Code corresponde a uma rotação dos participantes por diferentes tecnologias presentes na área de Computação. Durante o período do programa, o estagiário passa por cinco áreas distintas, obtendo um aprendizado de linguagens de programação como (i) Java e (ii) Ruby, (iii) do \textit{framework} .NET, (iv) de conceitos de garantia de qualidade de software, e (v) de conceitos e \textit{frameworks} para o desenvolvimento de interfaces de usuário (\textit{user interface}, UI) e experiência de usuário (\textit{user experience}, UX). O estudante exercita o aprendizado de cada área por meio de projetos internos da empresa, e apresenta um \textit{workshop} com o conteúdo aprendido ao final de cada etapa.

O objetivo desse trabalho é relatar as experiências do discente nas áreas em que o mesmo participou durante o período do Estágio Supervisionado, que correspondem ao aprendizado de Java e de conceitos de UI e UX. O aprendizado em cada uma das áreas é apresentado por meio do processo de desenvolvimento de um migrador de banco de dados em Java e da construção de interfaces de usuário por meio de \textit{frameworks} de UI e ferramentas de UX.

O restante desse trabalho está organizado de forma que o Capítulo~\ref{cap:estagio-supervisionado} apresenta a empresa e o seu Programa de Estágio, enquanto o Capítulo~\ref{cap:fundamentacao-teorica} aponta conceitos básicos de Java e de UI e UX. O Capítulo~\ref{cap:atividades-desenvolvidas} detalha as atividades em cada uma das áreas, enquanto o Capítulo~\ref{cap:conclusao} conclui o trabalho.
