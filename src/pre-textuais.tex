\imprimircapa
\imprimirfolhaderosto*

%\begin{fichacatalografica}
%\end{fichacatalografica}

%\begin{folhadeaprovacao}
%	\begin{center}
%		Espaço destinado à folha de aprovação
%	\end{center}
%\end{folhadeaprovacao}

%\begin{dedicatoria}
%	\vspace*{\fill}
%	\centering
%	\noindent
%	\textit{Dedicatória a ser escrita.} \vspace*{\fill}
%\end{dedicatoria}

%\begin{agradecimentos}
%	Agradecimentos a serem escritos.
%\end{agradecimentos}

%\setlength{\absparsep}{18pt}

\begin{resumo}
	Este trabalho descreve as atividades desenvolvidas durante o período de Estágio Supervisionado do aluno Pedro Felipe Froes do curso de Engenharia de Computação do CEFET-MG enquanto estagiário na empresa Avenue Code. Parte do programa de Jedi Internship da empresa é descrita, focando nas áreas de aprendizador Java e desenvolvimento de interface de usuários (UI). Ambas as áreas têm sua fundamentação teórica presente no Capítulo~\ref{cap:fundamentacao-teorica} deste trabalho, e o Capítulo~\ref{cap:atividades-desenvolvidas} descreve as atividades desenvolvidas em cada uma delas. Finalmente, o Capítulo~\ref{cap:conclusao} conclui o trabalho.
	
	\textbf{Palavras-chave}: Estágio supervisionado. Java. \textit{User interface}.
\end{resumo}

%\begin{resumo}[Abstract]
%	\begin{otherlanguage*}{english}
%    	Abstract to be written.
%		\textbf{Keywords}: keyword.
%	\end{otherlanguage*}
%\end{resumo}

\pdfbookmark[0]{\listfigurename}{lof}
\listoffigures*
\cleardoublepage

%\pdfbookmark[0]{\listtablename}{lot}
%\listoftables*
%\cleardoublepage

\begin{siglas}
	\item[API] Interface de programação de aplicações (\textit{application programming interface})
	\item[CSS] Folhas de estilo em cascata (\textit{cascading stylesheets})
	\item[DAO] Objeto de acesso a dados (\textit{data access object})
	\item[JDBC] Java Database Connectivity
	\item[HTML] Linguagem de marcação de hipertexto (\textit{hyper text markup language})
	\item[UI] Interfaces de usuário (\textit{user interfaces})
	\item[WORA] Escreva uma vez, rode em qualquer lugar (\textit{Write once, run anywhere})
\end{siglas}

\pdfbookmark[0]{\contentsname}{toc}
\tableofcontents*
%\cleardoublepage
