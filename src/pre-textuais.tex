\imprimircapa
\imprimirfolhaderosto*

%\begin{fichacatalografica}
%\end{fichacatalografica}

%\begin{folhadeaprovacao}
%	\begin{center}
%		Espaço destinado à folha de aprovação
%	\end{center}
%\end{folhadeaprovacao}

%\begin{dedicatoria}
%	\vspace*{\fill}
%	\centering
%	\noindent
%	\textit{Dedicatória a ser escrita.} \vspace*{\fill}
%\end{dedicatoria}

%\begin{agradecimentos}
%	Agradecimentos a serem escritos.
%\end{agradecimentos}

%\setlength{\absparsep}{18pt}

%\begin{resumo}
%	Resumo a ser escrito.
%	\textbf{Palavras-chave}: Meta-design. \textit{Design system}. Linguagem de domínio específico. DSL.
%\end{resumo}

%\begin{resumo}[Abstract]
%	\begin{otherlanguage*}{english}
%    	Abstract to be written.
%		\textbf{Keywords}: Meta-design. Design system. Domain-specific language. DSL.
%	\end{otherlanguage*}
%\end{resumo}

%\pdfbookmark[0]{\listfigurename}{lof}
%\listoffigures*
%\cleardoublepage

%\pdfbookmark[0]{\listtablename}{lot}
%\listoftables*
%\cleardoublepage

%\begin{siglas}
%	\item[CAD] Desenho Assistido por Computador (\textit{Computer Aided Design})
%	\item[CSS] Folhas de Estilo em Cascata (\textit{Cascading Style Sheets})
%	\item[DSL] Linguagem de domínio específico (\textit{domain specific language})
%	\item[GPL] Linguagem de propósito geral (\textit{general purpose language})
%	\item[HTML] Linguagem de Marcação de Hipertexto (\textit{Hyper Text Markup Language})
%\end{siglas}

\pdfbookmark[0]{\contentsname}{toc}
\tableofcontents*
\cleardoublepage
