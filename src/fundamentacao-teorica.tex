\chapter{Fundamentação teórica}
\label{cap:fundamentacao-teorica}

\section{Java}
\label{sec:java}

\section{\textit{User interface} e \textit{user experience}}
\label{sec:user-interface-e-user-experience}

%A apresentação de conceitos básicos pertinentes ao relatório é realizada neste capítulo. As definições de \textit{design system} e metadesign são mostradas, abordando também o processo de construção de um produto em design gráfico. Posteriormente, o conceito de linguagem de domínio específico é apresentado, apontando tanto o processo de desenvolvimento de uma linguagem quanto os benefícios oriundos de sua utilização em projetos.
%
%O designer gráfico assimila e representa conceitos verbais por meio de formas~\cite{madsen-2016}. O processo tradicional de realização dessa tarefa consiste na seleção e disposição de elementos como imagens, tipografia e cor pelo designer, seguido de um refinamento progressivo dessa disposição. O produto final é obtido quando o designer atende ao objetivo de comunicar a ideia pretendida de uma maneira considerada esteticamente boa.
%
%Se antes esse processo de assimilação era realizado manualmente com o produto final sendo publicado em meio impresso, hoje ele é desenvolvido principalmente por meio de ferramentas de desenho assistido por computador (\textit{computer aided design}, CAD), com seu resultado sendo difundido digitalmente. Embora tenha transicionado do meio impresso para o digital, a tarefa de produzir design gráfico continua essencialmente enraizada no processo tradicional. É valido notar que esse processo pode consumir uma grande quantidade de tempo, até mesmo para profissionais experientes~\cite{samara-2014}.
%
%\citeonline{gerstner-2007} propõe uma abordagem diferente para esse processo, apontando que a descrição do problema de assimilar conceitos em um produto também faz parte de sua solução. Em sua abordagem, o designer descreve imagens, tipografia, cor e outros elementos visuais que o produto deve ter, e um sistema produz possíveis soluções com diferentes disposições dos elementos descritos. Ao conjunto correspondente tanto à elaboração dessa descrição quanto à geração de produtos dá-se o nome de \textit{design system}, e ele possibilita que o processo tradicional de criação seja essencialmente transformado em um processo de descrição e seleção~\cite{madsen-2016, gerstner-2007}.
%
%O conjunto de regras e instruções definidos em um \textit{design system} pode ser utilizado para gerar um ou mais produtos que pertençam à mesma identidade visual. Por exemplo, na criação de uma logomarca, um conjunto de instruções seria elaborado antes de sua criação em si, definindo fontes, cores e símbolos que sejam suficientes para identificá-la em diferentes meios e produtos~\cite{madsen-2016}. O \textit{design system} parte do princípio que se a instrução para desenhar algo for escrita de maneira geral e suficiente, a mesma instrução poderá ser utilizada para produzir formas semelhantes~\cite{knuth-1986}.
%
%Dessa forma, diferentemente de ser criado diretamente pelo profissional, em um \textit{design system} o produto é descrito para somente depois ser criado, sendo que o processo de descrevê-lo é conhecido como metadesign. Assim, a utilização de um \textit{design system} está atrelada tanto ao metadesign, por explicar como o design deve ser construído, quanto à geração procedural de design, por permitir que os padrões sejam utilizados na criação rápida de formas similares~\cite{madsen-2016}.
%
%\citeonline{knuth-1986}, no entanto, aponta que o metadesign é uma tarefa mais árdua que o design, dado que explicar como se desenha uma forma é mais trabalhoso do que de fato desenhá-la. Porém, mesmo que um tempo considerável seja consumido para formular uma descrição precisa, um \textit{design system} bem projetado possibilita que o designer teste variações de um produto rapidamente, visualizando possíveis protótipos para o mesmo. Portanto, o trabalho de descrever é compensado pelo tempo de gerar cada uma das variações manualmente e, além disso, a randomização pode ser utilizada no sistema para criar variações que o profissional não produziria se utilizasse a abordagem tradicional~\cite{madsen-2016}. 
%
%Entretanto, para fazer uso do metadesign, o designer deve descrever o seu produto em um \textit{design system} por meio de programação de computadores. Contudo, designers não necessariamente dominam a programação, sendo que a maioria dos cursos da área não abrangem a utilização de linguagens de programação como meio de elaboração de produtos. Portanto, reduzir a barreira entre designer e programação é uma das necessidades para facilitar a utilização do metadesign~\cite{madsen-2016} e, na área de Computação, uma linguagem de domínio específico (cujo conceito é apresentado na seção seção seguinte) pode ser utilizada como um integrador entre programação e usuários não habituados a programar.
%
%\section{Linguagem de domínio específico}
%\label{sec:linguagem-de-dominio-especifico}
%
%Muitas linguagens de programação são linguagens de domínio específico (\textit{domain specific language}, DSL) ao invés de linguagens de propósito geral (\textit{general purpose language}, GPL)~\cite{mernik-2005}. Entre DSLs amplamente utilizadas estão a Linguagem de Marcação de Hipertexto (\textit{Hyper Text Markup Language}, HTML) e as Folhas de Estilo em Cascata (\textit{Cascading Style Sheets}, CSS). Trocando a generalidade encontrada em GPLs por expressividade e facilidade de uso, uma DSL elimina ou reduz a necessidade de experiência em programação de computadores para sua compreensão. Dessa forma, sua utilização está aberta a um maior número de usuários, não se limitando aos desenvolvedores~\cite{gray-2008, mernik-2005}.
%
%DSLs aumentam o nível de abstração durante a descrição de um problema. Isso torna possível que um usuário com pouca experiência descreva programas e especificações de forma eficiente e rápida, considerando que ele já esteja habituado ao domínio de aplicação da linguagem~\cite{gray-2008}. Um código escrito por meio de uma DSL é usualmente conciso, legível, de simples entendimento e manutenção, propiciando facilidade na especificação de problemas com a linguagem~\cite{gray-2008, france-2005}.
%
%Entretanto, desenvolver uma DSL é uma tarefa que enfrenta dificuldades relacionadas tanto à pré-disposição do usuário em aprender uma nova linguagem quanto ao desenvolvimento da DSL em si. Dessa forma, o processo de desenvolvimento de uma DSL requer não só conhecimento do domínio em que a DSL será aplicada, mas também experiência em desenvolvimento de linguagens~\cite{gray-2008, mernik-2005}.
%
%\citeonline{mernik-2005} apontam quatro etapas consecutivas para guiar o desenvolvimento de uma DSL: decisão, especificação, projeto e implementação. A etapa de decisão busca delimitar o propósito da linguagem, enquanto a de especificação pode ser realizada de maneira formal ou informal, com o primeiro método recorrendo a expressões regulares, gramáticas e/ou autômatos para a definição de sua sintaxe. Optar por uma especificação formal é uma abordagem mais recomendada, dado que ela pode trazer à tona possíveis problemas do projeto, além de possibilitar a implementação da DSL por meio de ferramentas de desenvolvimento de linguagens.
%
%A terceira etapa no desenvolvimento de uma DSL é a etapa de projeto, que é caracterizada pela criação de uma nova linguagem ou pela exploração de uma já existente. Ao projetar sobre uma linguagem-base, pode-se adicionar funcionalidades a ela, restringir sua aplicação, ou realizar um \textit{piggyback}, isso é, utilizar aspectos da linguagem existente na nova DSL~\cite{mernik-2005}. Benefícios dessa prática incluem uma maior facilidade na etapa de implementação, bem como uma possível familiaridade prévia para com os usuários. Caso eles já possuam contato com a linguagem original, isso pode ser um fator contribuinte para contornar a relutância em aprender uma nova linguagem~\cite{france-2005, mernik-2005}.
%
%Finalmente, a etapa de implementação distingue sete abordagens diferentes, sendo que as principais envolvem o uso de um compilador, interpretador, pré-processador ou uma combinação entre eles para a tradução da DSL. Cada uma das abordagens descritas é recomendada de acordo com o propósito e projeto da nova linguagem~\cite{mernik-2005}.
%
%Independente da maneira como a linguagem é desenvolvida, DSLs bem-elaboradas apresentam simplicidade de entendimento, omissão de detalhes não relacionados ao domínio de aplicação e também completude, que permite uma ampla descrição de problemas e especificações dentro do domínio em questão~\cite{micallef-2015}. A DSL Gherkin~\footnote{A especificação do Gherkin no contexto do Cucumber está disponível em <https://github.com/cucumber/cucumber/wiki/Gherkin>.} , por exemplo, é utilizada na ferramenta de automação de testes Cucumber para habilitar a escrita de critérios de aceitação de software sem que o usuário se preocupe com a implementação dos testes; já pré-processadores de CSS fazem uso de DSLs como SASS e/ou LESS para extender as funcionalidades do CSS, que também é uma DSL em si~\cite{mazinanian-2016, micallef-2015}. Tanto o Gherkin quando o SASS e o LESS servem como exemplos de DSLs que obtém sucesso no contexto em que são aplicadas, servindo como motivação para a criação de uma DSL para o contexto do metadesign.