\chapter{Conclusão}
\label{cap:conclusao}

Por abranger dois espectros bem diferentes no desenvolvimento de sistemas, o período do Estágio Supervisionado foi certamente enriquecedor para o autor. Durante o período do Programa de Estágio na área de Java, exercitou-se diversos conceitos atrelados à orientação a objetos e às características dessa linguagem de programação, como o uso de encapsulamento dos dados e a separação de responsabilidades nas classes presentes no projeto. Além disso, trabalhar com uma API amplamente utilizada por programadores Java no desenvolvimento do \textit{backend} de sistemas dá ao autor a possibilidade de sair do básico da linguagem e aprofundar-se em conceitos mais avançados de Java.

O período de UI também propiciou o aprendizado de diversos conceitos no desenvolvimento de interfaces de usuário, desde os mais básicos relacionados ao HTML, CSS e JavaScript, quanto aos mais avançados atrelados à utilização do AngularJS. A utilização de diversas ferramentas disponibilizadas pelo \textit{framework}, como as diretivas e serviços, contribuíram também para um maior aprendizado no âmbito do desenvolvimento de sistemas Web. Ainda é válido ressaltar que tanto no período de UI quanto no período de Java, houve a preocupação com o desenvolvimento dos projetos seguindo boas práticas de programação, reutilizando código quando possível e obedecendo padrões de projeto consolidados.

O autor pode adquirir experiência tanto no desenvolvimento do \textit{backend} quanto no do \textit{frontend} de sistemas ao longo do período do Estágio Supervisionado. Além disso, as experiências de trabalho em equipe vividas e utilização da metodologia ágil ao longo do desenvolvimento são enriquecedoras para o perfil profissional do autor, que terminou o Estágio Supervisionado com um conhecimento mais amplo no desenvolvimento de sistemas na área de Computação.